\chapter{Aspectos administrativos}
\thispagestyle{empty}

\section{Cronograma de actividades}
\section{Presupuesto}
\section{Fuentes de financiamiento}

% \begin{itemize}[leftmargin=1em]
%     \item La siguiente Tabla \ref{tab:my-table1} es de clasificación de suelos:
% \end{itemize}
% \begin{table}[H]
% \centering
% \begin{threeparttable}
% \caption[Clasificación de suelos]{Clasificación de suelos para determinarlo en el laboratorio del departamento de ordenamiento territorial}
% \label{tab:my-table1}
% \begin{tabular}{@{}ccccc@{}}
% \hline
% \multicolumn{1}{|c|}{} &
%   \multicolumn{1}{c|}{BRITÁNICO} &
%   \multicolumn{1}{c|}{AASHTO} &
%   \multicolumn{1}{c|}{ASTM} &
%   \multicolumn{1}{c|}{SUCS} \\ \cline{2-5} 
% \multicolumn{1}{|c|}{\multirow{-2}{*}{SISTEMAS}} &
%   \multicolumn{1}{c|}{{\color[HTML]{4D5156} (mm)}} &
%   \multicolumn{1}{c|}{{\color[HTML]{4D5156} (mm)}} &
%   \multicolumn{1}{c|}{{\color[HTML]{4D5156} (mm)}} &
%   \multicolumn{1}{c|}{{\color[HTML]{4D5156} (mm)}} \\ \hline
% \multicolumn{1}{|c|}{Grava} &
%   \multicolumn{1}{c|}{60 - 2} &
%   \multicolumn{1}{c|}{75 - 2} &
%   \multicolumn{1}{c|}{\textgreater 2} &
%   \multicolumn{1}{c|}{75 - 4,75} \\ \hline
% \multicolumn{1}{|c|}{Arena} &
%   \multicolumn{1}{c|}{2 - 0,006} &
%   \multicolumn{1}{c|}{2 - 0,05} &
%   \multicolumn{1}{c|}{2 - 0,075} &
%   \multicolumn{1}{c|}{4,75 - 0,075} \\ \hline
% \multicolumn{1}{|c|}{Limo} &
%   \multicolumn{1}{c|}{0,06 - 0,002} &
%   \multicolumn{1}{c|}{0,05 - 0,002} &
%   \multicolumn{1}{c|}{0,075 - 0,005} &
%   \multicolumn{1}{c|}{\textless 0,075 FINOS} \\ \hline
% \multicolumn{1}{|c|}{Arcilla} &
%   \multicolumn{1}{c|}{\textless  0,002} &
%   \multicolumn{1}{c|}{\textless  0,002} &
%   \multicolumn{1}{c|}{\textless  0,005} &
%   \multicolumn{1}{c|}{} \\ \hline
% \end{tabular}
%     \begin{tablenotes}
%     \vspace{-0.5cm}
%       \item {{\fontsize{10pt}{ \baselineskip}\selectfont \textbf{FUENTE}: Elaboración propia}}
%     \end{tablenotes}
% \end{threeparttable}
% \end{table}


% \begin{table}[H]
%     \centering
%     \begin{threeparttable}
%         \caption{Estaciones meteorológicas \sep}
%         \label{tab:my-table}
%         \begin{tabular}{@{}lcccc@{}}
%             \hline
%             Estaciones & Este (m)  & Norte (m)  & Latitud                & Longitud           \\ \hline
%             Tunel cero & 490680.81 & 8534186.15 & 13$^{\circ}$15'33.54'' & 75$^{\circ}$5'8''  \\
%             Ocucaje    & 426464.86 & 8410316.41 & 14$^{\circ}$22'42.2''  & 75$^{\circ}$40'0'' \\ \hline
%         \end{tabular}
%         \begin{tablenotes}
%             \item {{\fontsize{10pt}{ \baselineskip}\selectfont \textit{Nota.} Elaboración propia}}
%         \end{tablenotes}
%     \end{threeparttable}
% \end{table}

% \begin{table}[H]
% \centering
%   \begin{threeparttable}
% \caption{Estaciones meteorológicas}
% \label{tab:my-table2}
% \begin{tabular}{@{}lcccc@{}}
% \hline
% Estaciones & Este (m) & Norte (m) & Latitud & Longitud \\ \hline
% Tunel cero & 490680.81 & 8534186.15 & 13$^{\circ}$15'33.54'' & 75$^{\circ}$5'8'' \\
% Ocucaje & 426464.86 & 8410316.41 & 14$^{\circ}$22'42.2'' & 75$^{\circ}$40'0'' \\ \hline
% \end{tabular}
%     \begin{tablenotes}
%     \vspace{-0.5cm}
%       \item {{\fontsize{10pt}{ \baselineskip}\selectfont \textbf{FUENTE}: Elaboración propia}}
%     \end{tablenotes}
% \end{threeparttable}
% \end{table}

% \begin{figure}[H]
%     \centering
%       \caption{Comparación entre Qdisponible y Qdemanda}
%         \includegraphics[width=0.7\textwidth]{Figures/Comparation.pdf}
%         \captionsetup{labelfont=rm,skip=2pt,textfont=rm,font=small}
%         \caption*{\textbf{FUENTE:} Elaboración propia}
%     \label{fig:12}
% \end{figure}

% \begin{itemize}
%     \item En la Figura \ref{fig:21}, se muestran los volúmenes medios mensuales acumulados de la estación Santa Eulalia (1972-1992).
% \end{itemize}

% \begin{figure}[H]
%     \centering
%     \includegraphics[width=0.7\textwidth]{Figures/Volumenes.pdf}
%     \caption{Volúmenes medios mensuales acumulados}
%     \captionsetup{labelfont=rm,skip=2pt,textfont=rm,font=small}
%         \caption*{\textbf{FUENTE:} Servicio Nacional de Meteorología e Hidrología del Perú SENAMHI}
%     \label{fig:21}
% \end{figure}

% \begin{table}[H]
% \centering
%   \begin{threeparttable}
%     \caption{Sample ANOVA table}
%      \begin{tabular}{lllll}
%         \toprule \toprule
%         Stubhead & \( df \) & \( f \) & \( \eta \) & \( p \) \\
%         \midrule
%                  &     \multicolumn{4}{c}{Spanning text}     \\
%         Row 1    & 1        & 0.67    & 0.55       & 0.41    \\
%         Row 2    & 2        & 0.02    & 0.01       & 0.39    \\
%         Row 3    & 3        & 0.15    & 0.33       & 0.34    \\
%         Row 4    & 4        & 1.00    & 0.76       & 0.54    \\
%         \bottomrule
%      \end{tabular}
%     \begin{tablenotes}
%     \vspace{-0.5cm}
%         \item {{\fontsize{10pt}{ \baselineskip}\selectfont \textbf{FUENTE}: Elaboración propia}}
%     \end{tablenotes}
% \end{threeparttable}
% \end{table}




